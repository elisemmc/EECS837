\documentclass[10pt]{amsart}
\usepackage[letterpaper, landscape, margin=0.5in]{geometry}                % See geometry.pdf to learn the layout options. There are lots.               % ... or a4paper or a5paper or ... 
%\geometry{landscape}                % Activate for for rotated page geometry
\usepackage[parfill]{parskip}    % Activate to begin paragraphs with an empty line rather than an indent
\usepackage{graphicx}
\usepackage{amssymb}
\usepackage{epstopdf}
\usepackage[normalem]{ulem}
\DeclareGraphicsRule{.tif}{png}{.png}{`convert #1 `dirname #1`/`basename #1 .tif`.png}
\usepackage{array}
\usepackage{xcolor}

\newcommand{\hl}[1]{%
  \colorbox{red!50}{$\displaystyle#1$}}

\title{EECS 837 Homework 2}
\author{Elise McEllhiney}
\date{\today}                                           % Activate to display a given date or no date

\begin{document}
\maketitle

\begin{center}
\begin{tabular}{ | c | c | c | c | c | c | c | c | c |}
\hline
 & \multicolumn{3}{|c|}{Attributes} & Decision \\
 \hline
 & Wind & Humidity & Temperature & Trip \\
 \hline
 1 & 5 & 20 & 26 & yes\\ 
 2 & 10 & 40 & 20 & yes\\  
 3 & 5 & 60 & 20 & yes\\  
 4 & 10 & 20 & 16 & no\\ 
 5 & 15 & 40 & 20 & no\\ 
 6 & 10 & 50 & 26 & no\\ 
 \hline 
\end{tabular}
\end{center}

\section{Discretize the following decision table using.}
\subsection{a) Dominant Attribute approach based on conditional entropy}
\subsection{b) Multiple Scanning approach based on conditional entropy (a single scan)}
\subsection{c) Globalized Equal Frequency per Interval method}




\end{document}  